\documentclass [landscape,8pt,a4paper,twocolumn]{article}
\title {NCTU\_TaNoShiI Codebook}
\usepackage{parskip}
\usepackage{xeCJK} 
\setCJKmainfont{SourceHanSans-Regular}
\setmonofont{Monaco}
\usepackage {listings}
\usepackage {color}
\usepackage [left=1.0cm, right=1.0cm, top=2.0cm, bottom=0.5cm]{geometry}
\definecolor {mygreen}{rgb}{0,0.6,0}
\definecolor {mygray}{rgb}{0.5,0.5,0.5}
\definecolor {mymauve}{rgb}{0.58,0,0.82}
\usepackage{fancyheadings}
\rhead{\thepage}
\chead{初始化?陣列大小?\texttt{x, y}沒寫反?爆\texttt{int}?1-based?好,傳囉!}
\lhead{NCTU\_TaNoShiI}
\pagestyle{fancy}
\cfoot{}
\setlength{\headsep}{5pt}
\setlength{\textheight}{540pt}

\lstset {
  backgroundcolor=\color{white},
  basicstyle=\footnotesize\ttfamily,
  breakatwhitespace=false,
  breaklines=true,
  captionpos=b,
  commentstyle=\color{mygreen},
  deletekeywords={...},
  escapeinside={\%*}{*)},
  extendedchars=true,
  frame=single,
  keepspaces=true,
  keywordstyle=\color{blue},
  language=Octave,
  morekeywords={*,...},
  numbers=left,
  numbersep=4pt,
  numberstyle=\scriptsize\ttfamily\color{mygray},
  rulecolor=\color{black},
  showspaces=false,
  showstringspaces=false,
  showtabs=false,
  stepnumber=1,
  stringstyle=\color{mymauve},
  tabsize=2,
  xleftmargin=15pt,
  framexleftmargin=15pt,
  framexrightmargin=0pt,
  framexbottommargin=0pt,
  framextopmargin=0pt,
}

\begin {document}
\thispagestyle{fancy}
{ \Huge NCTU\_TaNoShiI}
\tableofcontents

\section{String}
\subsection{KMP}
\lstinputlisting [language=c++] {"code/String/KMP.cpp"}
\subsection{AC Automaton}
\lstinputlisting [language=c++] {"code/String/AC_Automaton.cpp"}
\subsection{Suffix Array}
\lstinputlisting [language=c++] {"code/String/Suffix_Array.cpp"}
\subsection{BWT}
\lstinputlisting [language=c++] {"code/String/BWT.cpp"}
\subsection{Suffix Automaton}
\lstinputlisting [language=c++] {"code/String/Suffix_Automaton.cpp"}
\subsection{Z Algorithm}
\lstinputlisting [language=c++] {"code/String/Z_Algorithm.cpp"}
\section{Convolution}
\subsection{FFT}
\lstinputlisting [language=c++] {"code/Convolution/FFT.cpp"}
\section{Java}
\subsection{Big Integer}
\lstinputlisting [language=java] {"code/Java/Big Integer.java"}
\subsection{Prime}
\lstinputlisting [language=java] {"code/Java/Prime.java"}
\section{Geometry}
\subsection{Fermat's Point}
\lstinputlisting [language=c++] {"code/Geometry/Fermat's Point.cpp"}
\subsection{Half Plane Intersection}
\lstinputlisting [language=c++] {"code/Geometry/Half_plane_Intersection.cpp"}
\subsection{Minimum Covering Circle}
\lstinputlisting [language=c++] {"code/Geometry/Minimum_Covering_Circle.cpp"}
\subsection{Geometry}
\lstinputlisting [language=c++] {"code/Geometry/Geometry.cpp"}
\subsection{K-closet Pair}
\lstinputlisting [language=c++] {"code/Geometry/K-closet Pair.cpp"}
\section{Graph}
\subsection{SCC (Tarjan)}
\lstinputlisting [language=c++] {"code/Graph/SCC_(Tarjan).cpp"}
\subsection{Maximun Clique}
\lstinputlisting [language=c++] {"code/Graph/maximun_clique.cpp"}
\subsection{2-SAT}
\lstinputlisting [language=c++] {"code/Graph/2-SAT.cpp"}
\subsection{Heavy Light Decomposition}
\lstinputlisting [language=c++] {"code/Graph/Heavy_Light_Decomposition.cpp"}
\subsection{SCC (Kosaraju)}
\lstinputlisting [language=c++] {"code/Graph/SCC_(Kosaraju).cpp"}
\subsection{Articulation Point}
\lstinputlisting [language=c++] {"code/Graph/Articulation Point.cpp"}
\subsection{BCC}
\lstinputlisting [language=c++] {"code/Graph/BCC.cpp"}
\section{Data Structure}
\subsection{K-D Tree (Insert)}
\lstinputlisting [language=c++] {"code/Data Structure/K-D Tree (Insert).cpp"}
\subsection{Treap}
\lstinputlisting [language=c++] {"code/Data Structure/Treap.cpp"}
\subsection{Segment Tree (Lazy)}
\lstinputlisting [language=c++] {"code/Data Structure/segment_tree (Lazy).cpp"}
\section{Matching}
\subsection{Stable Marriage}
\lstinputlisting [language=c++] {"code/Matching/Stable Marriage.cpp"}
\subsection{Blossom}
\lstinputlisting [language=c++] {"code/Matching/Blossom.cpp"}
\subsection{Min Cost Flow}
\lstinputlisting [language=c++] {"code/Matching/Min Cost Flow.cpp"}
\subsection{Dinic}
\lstinputlisting [language=c++] {"code/Matching/Dinic.cpp"}
\subsection{KM}
\lstinputlisting [language=c++] {"code/Matching/KM.cpp"}
\subsection{Bipartite Matching}
\lstinputlisting [language=c++] {"code/Matching/Bipartite Matching.cpp"}
\section{Mathematics}
\subsection{Extended GCD}
\lstinputlisting [language=c++] {"code/Mathematics/Extended GCD.cpp"}
\subsection{Sprague-Grundy}
\lstinputlisting [language=c++] {"code/Mathematics/Sprague-Grundy.cpp"}
\subsection{Lucas's Theorem}
\lstinputlisting [language=python] {"code/Mathematics/Lucas's Theorem.py"}
\subsection{Pollard's Rho Algorithm}
\lstinputlisting [language=c++] {"code/Mathematics/Pollard's rho algorithm.cpp"}
\subsection{Gauss-Jordan Elimination}
\lstinputlisting [language=c++] {"code/Mathematics/Gauss-Jordan_Elimination.cpp"}
\subsection{Miller-Rabin}
\lstinputlisting [language=c++] {"code/Mathematics/Miller-Rabin.cpp"}
\section{Building Environment}
\subsection{Vimrc}
\lstinputlisting [language=bash] {"code/Building Environment/vimrc"}
\subsection{Print}
\lstinputlisting [language=bash] {"code/Building Environment/print.sh"}

\section{monge}
$i \leq i^{'} < j \leq j^{'}$ \\
$m(i,j)+m(i^{'},j^{'}) \leq m(i^{'},j)+m(i,j^{'})$ \\
$k(i,j-1)<=k(i,j)<=k(i+1,j)$

\section{四心}
$\frac{sa*A+sb*B+sc*C}{sa+sb+sc}$ \\
外心 sin 2A : sin 2B : sin 2C \\
內心 sin  A : sin  B : sin  C \\
垂心 tan  A : tan  B : tan  C \\
重心      1 :      1 :      1 

\section{Runge-Kutta}
$y_{n+1}=y_n+\frac{h}{6}(k_1+2k_2+2k_3+k_4)$\\
$k_1=f(t_n,y_n)$\\
$k_2=f(t_n+\frac{h}{2},y_n+\frac{h}{2}k_2)$\\
$k_3=f(t_n+\frac{h}{2},y_n+\frac{h}{2}k_3)$\\
$k_2=f(t_n+h,y_n+hk_3)$

\section{Householder Matrix}
$I-2\frac{vv^T}{v^Tv}$

\section{Simpson's-rule}
$\int_a^bf(x)dx=\frac{b-a}{6}(f(a)+4f(\frac{a+b}{2})+f(b))$

\end{document}

